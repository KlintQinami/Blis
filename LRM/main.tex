\documentclass[11pt]{article}
\usepackage[letterpaper, portrait, margin=1.5in]{geometry}
\usepackage[utf8]{inputenc}
\usepackage{amsmath, amsthm, amssymb}
\usepackage{listings}
\usepackage{tkz-graph}  
\usepackage{mathtools}
\usepackage{multicol}

\newcommand{\code}[1]{\texttt{#1}}


\definecolor{dkgreen}{rgb}{0,0.6,0}
\definecolor{gray}{rgb}{0.5,0.5,0.5}
\definecolor{mauve}{rgb}{0.58,0,0.82}

\lstset{frame=tb,
  language=C,
  aboveskip=3mm,
  belowskip=3mm,
  showstringspaces=true,
  columns=flexible,
  basicstyle={\small\ttfamily},
  numbers=none,
  numberstyle=\tiny\color{gray},
  keywordstyle=\color{blue},
  commentstyle=\color{dkgreen},
  stringstyle=\color{mauve},
  breaklines=true,
  breakatwhitespace=true,
  tabsize=4
}

\title{Blis: Better Language for Image Stuff \\
  \Large Language Reference Manual \\
\Large Programming Languages and Translators, Spring 2017 }


\author{
  Abbott, Connor (cwa2112) \hspace{5mm} \hfill \texttt{[System Architect]} \\
  Pan, Wendy (wp2213)                   \hfill \texttt{[Manager]}          \\
  Qinami, Klint (kq2129)                \hfill \texttt{[Language Guru]}    \\
  Vaccaro, Jason (jhv2111)               \hfill \texttt{[Tester]}       
}


\begin{document}

\maketitle

\section{Introduction}

    Blis is a programming language for writing
    %
    hardware-accelerated 3D rendering programs. 
    %
    It gives the programmer fine-grained control of the 
    %
    graphics pipeline while abstracting away the burdensome 
    %
    details of OpenGL. The Blis philosophy is that OpenGL 
    %
    provides more power and flexibility than some graphics 
    %
    programmers will ever need. These OpenGL programmers are 
    %
    forced to write boilerplate code and painfully long 
    %
    programs to accomplish the simplest of tasks. By exposing %
    only the bare essentials of the graphics pipeline to the 
    %
    programmer,  Blis decreases the number of decisions that 
    %
    a graphics developer has to make. The result is sleek, 
    %
    novice-friendly code.
    %
    
    
    With Blis, you can write real-time 3D rendering programs 
    %
    such as 3D model viewers. You can write shadow maps for 
    %
    rendering dynamic shadows and use render-to-texture 
    %
    techniques to produce a variety of effects. In short, the 
    %
    idea is that you can write programs that manipulate Blis’
    %
    simplified graphics pipeline.
    
    
    In particular, writing vertex and fragment shaders is now more convenient. 
    %
    Rather than having to write shader code in a separate shader language 
    %
    (GLSL), you can use Blis to write both tasks that run on the GPU and those 
    %
    that run on the CPU. Consequently, shaders can reuse code from other parts 
    %
    of the program. Uniforms, attributes, and textures are registered with 
    %
    shaders by simply passing them as arguments into user-defined shader 
    %
    functions. Furthermore, loading shaders from the CPU to GPU is easily 
    %
    accomplished by uploading to a built-in pipeline object. See the “Parts of 
    %
    the Language” section below for more information about this pipeline object, %
    which is a central feature of Blis. 
    
    Blis has two backends: one for compiling source code to LLVM and another for %
    compiling to GLSL. The generated LLVM code links to the OpenGL library in 
    %
    order to make calls that access the GPU. 


\section{Sorts of programs to be written}
    Rendering is a major subtopic of 3D computer graphics, with many active 
    %
    researchers developing novel techniques for light transport modeling. These 
    %
    researchers would like to test their new ray tracing, ray casting, and 
    %
    rasterization developments by actual experimentation. These new methods would be 
    %
    translated into rendering programs written in our language. Because these 
    %
    programs would be written at a higher level of abstraction than OpenGL, they 
    %
    would resemble mathematical thinking and derivations. Our language would also 
    %
    facilitate users to make larger tweaks to their programs with fewer lines of 
    %
    code, allowing for a more effective use of their time in experimentation. 
    %
    Minimizing this transaction cost between research and code should allow for more 
    %
    development of rendering software overall.

\iffalse
\section{Parts of the language and what they do}
    \begin{itemize}
        \item Data types: Integers (int), floating point numbers (float), single and multidimensional arrays, matrices (up to 4x4) (matrix), strings (string) -- for file I/O and writing
        \item  Built-in cos, sin, exp, and log functions
        \item  Loops (for and while), if and else statements
        \item  User-defined functions
        \item  Comments
        \item  Buffer/Image opaque types for data on the GPU
            \begin{itemize}
            \item Represent data stored on the GPU
            \item Only way to interact is through uploading/downloading to arrays
            \end{itemize}
        \item Shaders
            \begin{itemize}
            \item Functions annotated for use on the GPU
            \item Arguments represent resources 
            \end{itemize}
        \item Pipeline built-in type for actual rendering
            \begin{itemize}
                \item  Can “bind” shaders, buffers and images to the pipeline for use as inputs/outputs
            \end{itemize}
    \end{itemize}
\fi

\section{Lexical Conventions}

For the most part, tokens in Blis are similar to C. The main difference is the extra keywords added to support vector and matrix types. There are four kinds of tokens: identifiers, keywords, literals, and expression operators like \code{(}, \code{)}, and \code{*}. Blis is a free-format language, so comments and whitespace are ignored except to separate tokens.

\subsection{Comments}

Blis has C-style comments, beginning with \code{/*} and ending with \code{*/} which do not nest.

\subsection{Identifiers}

An identifier consists of an alphabetic character or underscore followed by a sequence of letters, digits, and underscores.

\subsection{Keywords}

The following identifiers are reserved as keywords, and may not be used as identifiers:

\begin{multicols}{3}
\begin{itemize}
	\item \code{int}
	\item \code{char}
	\item \code{float}
	\item \code{bool}
	\item \code{struct}
	\item \code{return}
	\item \code{break}
	\item \code{continue}
	\item \code{if}
	\item \code{else}
	\item \code{for}
	\item \code{while}
	\item \code{true}
	\item \code{false}
	\item \code{vec2}
	\item \code{vec3}
	\item \code{vec4}
	\item \code{mat2x2}
	\item \code{mat2x3}
	\item \code{mat2x4}
	\item \code{mat3x2}
	\item \code{mat3x3}
	\item \code{mat3x4}
	\item \code{mat4x2}
	\item \code{mat4x3}
	\item \code{mat4x4}
	\item \code{bvec2}
	\item \code{bvec3}
	\item \code{bvec4}
	\item \code{bmat2x2}
	\item \code{bmat2x3}
	\item \code{bmat2x4}
	\item \code{bmat3x2}
	\item \code{bmat3x3}
	\item \code{bmat3x4}
	\item \code{bmat4x2}
	\item \code{bmat4x3}
	\item \code{bmat4x4}
	\item \code{RGBA8}
	\item \code{RGB8}
	\item \code{RGBA8}
	\item \code{RG8}
	\item \code{RGBA8}
	\item \code{R8}
	\item \code{RGBA8}
	\item \code{@vertex}
	\item \code{@fragment}

\end{itemize}
\end{multicols}

\subsection{Literals}

\subsubsection{Integer Literals}

Integer literals consist of a sequence of one or more of the digits 0-9. Constants must be in decimal.

\subsubsection{Floating Point Literals}

A floating-point literal consists of an integer part, a decimal, a fractional part, and \code{e} followed by an optionally signed exponent. Both the integer and fractional parts consist of a sequence of digits. At least one of the integer and fractional parts must be present, and at least one of the decimal point and the exponent must be present. Since Blis only supports single-precision floating point, all floating point constants are considered single-precision.

\subsubsection{Character Literals} \label{charliterals}

A character literal consists of a single character, or a backslash followed by one of \code{'}, \code{n}, \code{r}, \code{t}, or a sequence of decimal digits that must form a number from 0 to 255 which represents an ASCII code. \code{'\textbackslash''}, \code{'\textbackslash{}n'}, \code{'\textbackslash{}r'}, and \code{'\textbackslash{}t'} have the usual meanings.

\subsubsection{Boolean Literals}

Boolean literals consist of the two keywords \code{true} and \code{false}.

\subsubsection{String Literals}

String literals consist of a series of characters surrounded by \code{"}. All the escapes described in section~\ref{charliterals} are supported, as well as \code{\textbackslash{}"} to escape a \code{"}. String literals have type \code{char[n]}, i.e. a fixed-size array of characters where \code{n} is the number of characters. For example, \code{"Hello world"} has type \code{char[11]}. Unlike C, there is no extra \code{'\textbackslash{}0'} inserted at the end of the string.

\section{Syntax and Semantics}

\subsection{Basic Types}

\subsubsection{Integers}

Blis supports integers through the \code{int} type, which is a 32-bit two's-complement signed integer.

\subsubsection{Floating Point}

Blis supports single-precision (32-bit) floating point numbers through the \code{float} type.

\subsubsection{Booleans}

Blis supports booleans through the \code{bool} type. Booleans may only ever be \code{true} or \code{false}.

\subsection{Vectors and Matrices}

To more easily represent position data and color data, Blis supports vectors and matrices of floating-point numbers with the \code{vec2}, \code{vec3}, and \code{vec4} builtin types, as well as \code{matNxM} where  $1 \leq \code{N} \leq 4$ and $1 \leq \code{M} \leq 4$, similar to GLSL. Blis supports the full complement of matrix-vector multiplication operations on these types as described in section~\ref{operators}, as well as component-wise multiplication and a number of built-in functions described in section~\ref{builtins}.

Blis also supports vectors and matrices of boolean type, denoted by \code{bvec2}, \code{bmat2x2}, etc., which are the result of the comparison operators \code{==}, \code{!=}, \code{>}, \code{<}, \code{>=}, and \code{<=} applied to floating-point vectors and matrices. The builtin functions \code{all()} and \code{any()} can be used to reduce these types to a single boolean value, so that e.g. \code{all(foo == bar)} checks that all the components of \code{foo} and \code{bar} are equal.

In addition to the special operators and built-in functions, \code{vecN} types have \code{x}, \code{y}, \code{z}, and \code{w} members as appropriate as if they were normal structures, and \code{matNxM} types have \code{xx}, \code{xy}, \code{yx}, etc. Thus, the syntax to access the \code{x} component of a \code{vec4 foo} is \code{foo.x}. This also applies to the boolean vector and matrix types.
\subsection{Arrays}

\subsection{Structures}

\subsection{Constants}

\subsection{Type Conversions}

\subsection{Operators} \label{operators}

\subsection{Functions}

\subsection{Builtin Functions} \label{builtins}

\subsection{Buffers and Images}

\subsection{Buffer and Image Formats}

\subsection{Samplers}

\subsection{Framebuffers}

\subsection{Pipelines}

\section{Formal Grammar}

\section{Sample Code}

\begin{lstlisting}
/* Draws a spinning triangle */

var transform: mat3;

fn generate_transform(angle: float) {
    transform = mat3(cos(angle), -sin(angle), 0,
               sin(angle), cos(angle), 0,
               0, 0, 1);
}

/* This creates a pipeline, which includes everything necessary
 * to draw some triangles to the framebuffer(s).
 */
Pipeline {
    /* The vertex shader is responsible for transforming
     * triangle vertices into screen space. "color" and
     * "pos" are per-vertex inputs, called "attributes" in
     * OpenGL. "color" is a per-vertex output that gets
     * interpolated across the triangle, called a "varying"
     * in OpenGL, while "position" is a builtin that 
     * determines where the triangle is rendered.
     */
    vertex_shader: fn @vertex (color: vec3, pos: vec3) ->
    	(color: vec4, position: vec4) {
    	/* Note that here, we're pulling in "transform", which was
    	 * declared as a global variable outside the pipeline. This
    	 * is a simple way of passing data that doesn't change per-vertex
    	 * to the shader. In OpenGL, this would require creating a
    	 * uniform, binding it, and updating it per-draw if it changes.
    	 */
        return (vec4(color, /* alpha */ 1),
   				 vec4(transform * pos, /* w coordinate */ 1));
    },

    /* The fragment shader is responsible for determining
     * the color of the "fragment" (potential pixel).
     * "color" is a varying input, matched by name with
     * "color" from the vertex shader.
     */
    fragment_shader: fn @fragment (color: vec4) -> (framebuffer: vec4) {
        return color;
    },
    
    /* The "buffer" type is a way to represent data stored on the GPU.
     * Buffers cannot be accessed directly, they can only be uploaded
     * to or downloaded from. The type of the buffer, "vec3", means that
     * data is stored uncompressed in GPU memory as an array of vectors of
     * 3 floating point numbers (compare to the framebuffer below). This
     * particular buffer is used by the vertex shader as an attribute,
     * since the name matches the parameter of the function, which means
     * that assigning to this member will bind the buffer to the
     * Vertex Array Object (VAO) of the vertex shader in OpenGL terms.
     */
     color: Buffer<vec3>,
     
     /* similar to the above */
     pos: Buffer<vec3>,
    
    /* The framebuffer is what accumulates the final
     * color and depth. RGBA8 indicates that the pixels should
     * store their red, green, blue, and alpha as 8-bit integers
     * where 0 maps to 0.0 and 255 maps to 1.0, and is part of the
     * type.
     */
    framebuffer: Framebuffer<RGBA8>
} my_pipeline;

/* transfer data from the CPU to the GPU */
Buffer<vec3> color, pos;
color.upload([vec3(1.0, 0.0, 0.0),
			   vec3(0.0, 1.0, 0.0),
			   vec3(0.0, 0.0, 1.0)]);
pos.upload([vec3(-1.0, -1.0, 0.0),
			 vec3(1.0, -1.0, 0.0),
			 vec3(0.0,  1.0, 0.0)];
/* prepare the pipeline */
my_pipeline.color = color;
my_pipeline.pos = pos;

/* this creates a window and the framebuffer for it */
Window my_window = make_window(1024, 768, "My Window");

/* tell the pipeline to draw to the window's framebuffer */
my_pipeline.framebuffer = my_window.framebuffer;

/* main loop */
var angle: float = 0;
do {
    angle += 1.0;
    generate_transform(angle);
    draw(my_pipeline);
    swap_buffers(window);
} while (!should_close(window));

\end{lstlisting}

\end{document}

