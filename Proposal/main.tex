\documentclass[11pt]{article}
\usepackage[letterpaper, portrait, margin=1.5in]{geometry}
\usepackage[utf8]{inputenc}
\usepackage{amsmath, amsthm, amssymb}
\usepackage{amssymb}
\usepackage{mathtools}
\usepackage{listings}
\usepackage{mathabx}
\usepackage{tkz-graph}  
\usepackage{mathtools}


\definecolor{dkgreen}{rgb}{0,0.6,0}
\definecolor{gray}{rgb}{0.5,0.5,0.5}
\definecolor{mauve}{rgb}{0.58,0,0.82}

\lstset{frame=tb,
  language=C,
  aboveskip=3mm,
  belowskip=3mm,
  showstringspaces=true,
  columns=flexible,
  basicstyle={\small\ttfamily},
  numbers=none,
  numberstyle=\tiny\color{gray},
  keywordstyle=\color{blue},
  commentstyle=\color{dkgreen},
  stringstyle=\color{mauve},
  breaklines=true,
  breakatwhitespace=true,
  tabsize=4
}

\title{Blis: Better Language for Image Stuff \\
\Large Project Proposal \\
\Large Programming Languages and Translators, Spring 2017 }


\author{
  Abbott, Connor (cwa2112) \hspace{5mm} \hfill \texttt{[System Architect]} \\
  Pan, Wendy (wp2213)                   \hfill \texttt{[Manager]}          \\
  Qinami, Klint (kq2129)                \hfill \texttt{[Language Guru]}    \\
  Vaccaro, Jason (jhv2111)               \hfill \texttt{[Tester]}       
}


\begin{document}

\maketitle

\section{Introduction}

    Blis is a programming language for writing
    %
    hardware-accelerated 3D rendering programs. 
    %
    It gives the programmer fine-grained control of the 
    %
    graphics pipeline while abstracting away the burdensome 
    %
    details of OpenGL. The Blis philosophy is that OpenGL 
    %
    provides more power and flexibility than some graphics 
    %
    programmers will ever need. These OpenGL programmers are 
    %
    forced to write boilerplate code and painfully long 
    %
    programs to accomplish the simplest of tasks. By exposing %
    only the bare essentials of the graphics pipeline to the 
    %
    programmer,  Blis decreases the number of decisions that 
    %
    a graphics developer has to make. The result is sleek, 
    %
    novice-friendly code.
    %
    
    
    With Blis, you can write real-time 3D rendering programs 
    %
    such as 3D model viewers. You can write shadow maps for 
    %
    rendering dynamic shadows and use render-to-texture 
    %
    techniques to produce a variety of effects. In short, the 
    %
    idea is that you can write programs that manipulate Blis’
    %
    simplified graphics pipeline.
    
    
    In particular, writing vertex and fragment shaders is now more convenient. 
    %
    Rather than having to write shader code in a separate shader language 
    %
    (GLSL), you can use Blis to write both tasks that run on the GPU and those 
    %
    that run on the CPU. Consequently, shaders can reuse code from other parts 
    %
    of the program. Uniforms, attributes, and textures are registered with 
    %
    shaders by simply passing them as arguments into user-defined shader 
    %
    functions. Furthermore, loading shaders from the CPU to GPU is easily 
    %
    accomplished by uploading to a built-in pipeline object. See the “Parts of 
    %
    the Language” section below for more information about this pipeline object, %
    which is a central feature of Blis. 
    
    Blis has two backends: one for compiling source code to LLVM and another for %
    compiling to GLSL. The generated LLVM code links to the OpenGL library in 
    %
    order to make calls that access the GPU. 


\section{Sorts of programs to be written}
    Rendering is a major subtopic of 3D computer graphics, with many active 
    %
    researchers developing novel techniques for light transport modeling. These 
    %
    researchers would like to test their new ray tracing, ray casting, and 
    %
    rasterization developments by actual experimentation. These new methods would be 
    %
    translated into rendering programs written in our language. Because these 
    %
    programs would be written at a higher level of abstraction than OpenGL, they 
    %
    would resemble mathematical thinking and derivations. Our language would also 
    %
    facilitate users to make larger tweaks to their programs with fewer lines of 
    %
    code, allowing for a more effective use of their time in experimentation. 
    %
    Minimizing this transaction cost between research and code should allow for more 
    %
    development of rendering software overall.


\section{Parts of the language and what they do}
    \begin{itemize}
        \item Data types: Integers (int), floating point numbers (float), single and multidimensional arrays, matrices (up to 4x4) (matrix), strings (string) -- for file I/O and writing
        \item  Built-in cos, sin, exp, and log functions
        \item  Loops (for and while), if and else statements
        \item  User-defined functions
        \item  Comments
        \item  Buffer/Image opaque types for data on the GPU
            \begin{itemize}
            \item Represent data stored on the GPU
            \item Only way to interact is through uploading/downloading to arrays
            \end{itemize}
        \item Shaders
            \begin{itemize}
            \item Functions annotated for use on the GPU
            \item Arguments represent resources 
            \end{itemize}
        \item Pipeline built-in type for actual rendering
            \begin{itemize}
                \item  Can “bind” shaders, buffers and images to the pipeline for use as inputs/outputs
            \end{itemize}
    \end{itemize}

   
    

\section{Sample Code}

\begin{lstlisting}
/* Draws a spinning triangle */

var transform: mat3;

fn generate_transform(angle: float) {
    transform = mat3(cos(angle), -sin(angle), 0,
               sin(angle), cos(angle), 0,
               0, 0, 1);
}

/* This creates a pipeline, which includes everything necessary
 * to draw some triangles to the framebuffer(s).
 */
Pipeline {
    /* The vertex shader is responsible for transforming
     * triangle vertices into screen space. "color" and
     * "pos" are per-vertex inputs, called "attributes" in
     * OpenGL. "color" is a per-vertex output that gets
     * interpolated across the triangle, called a "varying"
     * in OpenGL, while "position" is a builtin that 
     * determines where the triangle is rendered.
     */
    vertex_shader: fn @vertex (color: vec3, pos: vec3) ->
    	(color: vec4, position: vec4) {
    	/* Note that here, we're pulling in "transform", which was
    	 * declared as a global variable outside the pipeline. This
    	 * is a simple way of passing data that doesn't change per-vertex
    	 * to the shader. In OpenGL, this would require creating a
    	 * uniform, binding it, and updating it per-draw if it changes.
    	 */
        return (vec4(color, /* alpha */ 1),
   				 vec4(transform * pos, /* w coordinate */ 1));
    },

    /* The fragment shader is responsible for determining
     * the color of the "fragment" (potential pixel).
     * "color" is a varying input, matched by name with
     * "color" from the vertex shader.
     */
    fragment_shader: fn @fragment (color: vec4) -> (framebuffer: vec4) {
        return color;
    },
    
    /* The "buffer" type is a way to represent data stored on the GPU.
     * Buffers cannot be accessed directly, they can only be uploaded
     * to or downloaded from. The type of the buffer, "vec3", means that
     * data is stored uncompressed in GPU memory as an array of vectors of
     * 3 floating point numbers (compare to the framebuffer below). This
     * particular buffer is used by the vertex shader as an attribute,
     * since the name matches the parameter of the function, which means
     * that assigning to this member will bind the buffer to the
     * Vertex Array Object (VAO) of the vertex shader in OpenGL terms.
     */
     color: Buffer<vec3>,
     
     /* similar to the above */
     pos: Buffer<vec3>,
    
    /* The framebuffer is what accumulates the final
     * color and depth. RGBA8 indicates that the pixels should
     * store their red, green, blue, and alpha as 8-bit integers
     * where 0 maps to 0.0 and 255 maps to 1.0, and is part of the
     * type.
     */
    framebuffer: Framebuffer<RGBA8>
} my_pipeline;

/* transfer data from the CPU to the GPU */
Buffer<vec3> color, pos;
color.upload([vec3(1.0, 0.0, 0.0),
			   vec3(0.0, 1.0, 0.0),
			   vec3(0.0, 0.0, 1.0)]);
pos.upload([vec3(-1.0, -1.0, 0.0),
			 vec3(1.0, -1.0, 0.0),
			 vec3(0.0,  1.0, 0.0)];
/* prepare the pipeline */
my_pipeline.color = color;
my_pipeline.pos = pos;

/* this creates a window and the framebuffer for it */
Window my_window = make_window(1024, 768, "My Window");

/* tell the pipeline to draw to the window's framebuffer */
my_pipeline.framebuffer = my_window.framebuffer;

/* main loop */
var angle: float = 0;
do {
    angle += 1.0;
    generate_transform(angle);
    draw(my_pipeline);
    swap_buffers(window);
} while (!should_close(window));

\end{lstlisting}

\end{document}

